\documentclass{article}
\usepackage{resume}

\begin{document}
\thispagestyle{firststyle} % bottom of first page will have a "continued..." footer
\small

\vspace*{-.5in} % Move up a bit to use more of the page

% Contact info header
\begin{center}
	{\LARGE \scshape {Eric Krause}}	\\
	8815 SW 36th Ave $\bullet$ Portland OR, 97219  $\bullet$ (541) 337-5788\\
	\texttt{\href{mailto:eric+resume@sauerkrause.org}{eric+resume@sauerkrause.org}}
  $\bullet$  \url{www.sauerkrause.org}\\
\end{center}


% Objective statement
%\header{Objective}
%	\hspace{-10pt} Seeking full-time engineering position in the fields of %Computer Architecture and/or Embedded Systems

% Skills section
\header{Technical Skills} 
	\vspace{-15pt}
	\begin{description}
		\item[Proficient Languages:] Verilog, C, Assembly (ARM, z80, MIPS, PicoBlaze)
		\item[Familiar Languages:] C++, SystemVerilog, Python, Bash, Java (for Android Development)
		\item[Hardware:] RTL design and debug, digital design and SoC/embedded system design and debug with FPGAs. Experience using test equipment in a laboratory setting to verify and debug digital designs.
		\item[Software:] Relatively OS agnostic; equal experience with Mac/Windows/Linux.  Professional experience with Xilinx Toolchain (ISE, Lab Tools, EDK, SDK).  Limited experience with GNU tools (Make, GCC) and VCS (git).
	\end{description}
	\vspace{-4pt}

% Coursework (Remove as needed)
\header{Relevant Coursework}
	\begin{table}[h!]
		\small
		\vspace{-19pt}
	\begin{tabular}{lll}
	$\bullet$ Microprocessor Design & $\bullet$ Superscalar Processor System Architecture & $\bullet$ Parallel Computing Architecture\\
	$\bullet$ SoC Design with FPGAs & $\bullet$ Embedded Systems with FPGAs & $\bullet$ Embedded Software Programming\\
	\end{tabular}
	\vspace{-19pt}
\end{table}	

% Education section
\header{Education}
	\hspace{-10pt}\schoolentry{M.S., Computer Engineering}{3.86}{Portland State University, Portland Oregon}{2011-2013}
  \vspace{3pt}

	\schoolentry{Post-Bac, Electrical/Computer Engineering}{3.88}{Portland State University, Portland Oregon}{2010-2012}
	\vspace{3pt}

	\schoolentry{B.A., Environmental Studies}{3.80}{University of Oregon, Eugene Oregon}{2005-2009}	

% Work History
\header{Work History}
	\hspace{-10pt}
	\textbf{Intel Corporation} Hillsboro, Oregon\\
	\emph{Data Center Group Intern -- 2011-2013}
	\begin{tight}
		\item RTL Development of an FPGA-based DDR3 Memory Error Injector (MEI) interposer. 
		\item Python utility development for automating testing of server platform error injection and verification using MEI.
		\item Optimization of FPGA RTL designs to improve timing and enhance features as per customer needs.
		\item Testing of runtime memory error injection and detection on Intel server platforms.
	\end{tight} 

	\textbf{Portland State University} Portland, Oregon\\
	\emph{IEEE Tutor (Computer Engineering) -- 2011-2013}		
	\begin{tight}
		\item Programming/Algorithms (Verilog, C, C++ Python)
		\item Digital Circuits (Logic Circuits, Boolean Algebra, Programmable Logic Devices, Simulation)
		\item Digital Systems (Synchronous Design, Timing Analysis, State Machines, FPGA Synthesis, Microprocessors)
	\end{tight}

	\textbf{Free Geek} Portland, Oregon\\
	\emph{Platform Deployment Tech (Volunteer) -- 2013-Current} 
	\begin{tight}
		\item Computer disassembly/assembly
		\item Peripherals test and debug
		\item OS/Software installation
	\end{tight}

% Awards, etc	
\header{Honors \& Awards}
	\vspace{-15pt}
	\begin{itemize}
		\item \NameUrlTextBlock{Intel Professional Recognition}{http://db.tt/0nqPo3Yf} 	\vspace{-16pt}
		\begin{itemize}
			\item Awarded for "excellent creative and technical abilities", after identifying, debugging and repairing a critical functionality in key product in 2013.
		\end{itemize} \vspace{-6pt}

		\item \NameUrlTextBlock{Ford Family Foundation Scholarships \emph{(Undergraduate and Graduate)}}{http://www.tfff.org/} \vspace{-16pt}
		\begin{itemize}
		\item One of fewer than 100 Scholars inducted into the prestigious Ford Family Foundation Undergraduate Scholarship program in 2005. \vspace{-3pt} 
		\item Awarded second, additional Graduate Scholarship following academic successes at University of Oregon and Portland State University. \vspace{-3pt}
		\end{itemize}
		\item \NameUrlTextBlock{Etta Kappa Nu (HKN)}{http://web.cecs.pdx.edu/~eta/} \vspace{-16pt}
		\begin{itemize}
			\item IEEE Honors Society. Limited to top 25\% of Department. \vspace{-3pt}
		\end{itemize}

		\item \NameUrlTextBlock{Golden Key International Honour Society}{https://www.goldenkey.org/}\vspace{-16pt}
		\begin{itemize}
			\item Member since 2008\vspace{-3pt}
		\end{itemize}

	\end{itemize}

% references
\header{References}
	\emph{Available upon request.}

% PAGE BREAK
\newpage
\fancyhf{}

% Projects	
\header{Academic Projects}
	\vspace{-15pt}
	
	\begin{itemize}
		\item \NameUrlTextBlock
		{FPGA-Based Color-Tracking Robot (Verilog, PicoBlaze)}{http://tinyurl.com/color-fpga-bot}\vspace{-16pt}
		\begin{itemize}
			\item 1$^{st}$ place in class competition for best term project (SoC design with FPGAs), awarded by course professors. \vspace{-3pt}
			\item Designed and built autonomous color-seeking robot controlled by SoC in FPGA.  \vspace{-3pt}
			\item Controlled via custom Verilog code and Picoblaze soft processor in FPGA of Spartan-6E development board
			\vspace{-3pt} 
			\item Computer vision via CMUCam4 camera and environmental awareness via proximity/light sensors.\vspace{-3pt}   		
			\end{itemize}
 
		\item \NameUrlTextBlock
		{Branch Predictor/BTB simulation (C)}	{https://github.com/rattboi/flanders_ece486}\vspace{-16pt}
		\begin{itemize}
			\item Accuracy of solution ranked within top 3 best designs in Computer Architecture course when tested against traces taken from unknown processor.\vspace{-3pt} 
			\item Development of branch predictor/branch target buffer simulation in C.\vspace{-3pt} 
			\item Design of N-way associative cache, Return Address Stack, and Fully-Associative cache simulation blocks in C++			
		\end{itemize}
		
		\item \NameUrlTextBlock
		{Microprocessor Cache Simulation (Verilog)}{https://github.com/ekrause/0xBEEFA55}\vspace{-16pt}
		\begin{itemize}
			\item Source code from this project has been incorporated into Microprocessor Design course materials by professor\vspace{-3pt} 
			\item Designed and coded split-level L1 data/instruction cache simulation in Verilog\vspace{-3pt}
			\item Project read in trace data from text file, performed cache simulation, and displayed hit/miss statistics.		\vspace{-3pt}
		\end{itemize}
		
	\end{itemize}
	\vspace{-4pt}

\end{document}
