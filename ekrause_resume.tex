\documentclass{article}
\usepackage{resume}
\usepackage[top=1in, bottom=.9in, left=.92in, right=.83in]{geometry}
\usepackage{setspace}
\setstretch{0.98}

\begin{document}
\fancyhf{}
%\small

\vspace*{-.5in} % Move up a bit to use more of the page

% Contact info header
\begin{center}
	{\LARGE \scshape {Eric Krause}}	\\
	Portland OR, 97219  $\bullet$ (541) 337-5788\\
	\texttt{\href{mailto:eric@sauerkrause.org}{eric@sauerkrause.org}}
  $\bullet$  \url{www.sauerkrause.org}\\
  %{\small \scshape Engineer I: Acme Co. [Job ID: 12345]}
\end{center}


% Education section
\header{Education}

	\hspace{-4pt}\schoolentry{M.S., Computer Engineering}{3.86}{Portland State University, Portland Oregon -- Expected Graduation: 12/2013}{2011-Current}

	\schoolentry{Post-Bac Studies, Electrical/Computer Engineering}{3.88}{Portland State University, Portland Oregon}{2010-2012}

	\schoolentry{B.A., Environmental Studies}{3.80}{University of Oregon, Eugene Oregon}{2005-2009}	
	
	
	\vspace{-5pt}


% Relavant Coursework 
\header{Relevant Coursework}
	\begin{table}[h!]\centering
	\small\vspace{-8pt}
	\begin{tabular}{lll}
	$\bullet$ Microprocessor Design & $\bullet$ Superscalar Processor System Architecture & $\bullet$ Parallel Computing Architecture\\
	$\bullet$ SoC Design with FPGAs & $\bullet$ Embedded Systems with FPGAs & $\bullet$ Embedded Software Programming\\
	\end{tabular}
	\vspace{-17pt}
\end{table}	


% Skills section
\header{Technical Skills} \vspace{-6pt}
	\begin{itemize}\setlength{\itemsep}{-4pt}

		\item\textbf{Proficient Languages:} Verilog, C, Assembly (ARM, z80, MIPS, PicoBlaze)
		\item\textbf{Familiar Languages:} C++, SystemVerilog, Python, Bash, Java (for Android Development)
		\item\textbf{Hardware:} RTL design and debug, digital design and SoC/embedded system design and debug with FPGAs.  Experience using test equipment in a laboratory setting to verify and debug digital designs.
		\item\textbf{Software:} Experience with Mac/Windows/Linux.  Professional experience with Xilinx Toolchain (ISE, Lab Tools, EDK, SDK).  Limited experience with GNU tools (Make, GCC) and VCS (git).
	\end{itemize}
	\vspace{-8pt}


% Work History
\header{Experience}\vspace{3pt}

\expheader{Data Center Group Intern}{Intel Corporation}{Hillsboro, Oregon}{3/2011 - 9/2013}

	\begin{itemize}\setlength{\itemsep}{-4pt}
		\item Wrote RTL and testbenches for an FPGA-based DDR3 Memory Error Injector (MEI) interposer.
		\item Developed automated testing CLI utilities in Python to simplify testing MEI on Intel server platforms.
		\item Optimized existing FPGA RTL designs to improve timing and enhance features per customer needs.
		\item Tested memory error injection functionality of MEI and memory error detection on Intel server platforms.
		\item Received professional award for identifying, debugging and repairing a bug affecting a critical function of a key product in 2013.

	\end{itemize} 

\expheader{IEEE Computer Engineering Tutor}{Portland State University}{Portland, Oregon}{9/2012 - Current}

	\begin{itemize}\setlength{\itemsep}{-4pt}
		\item Instructed students on a variety of Computer Engineering topics, including programming and digital design.
		\item Adapted teaching strategies to the unique background and abilities level of each student.
	\end{itemize}

\expheader{Platform Deployment Tech Volunteer}{Free Geek}{Portland, Oregon}{9/2013 - Current}
	
	\begin{itemize} \setlength{\itemsep}{-4pt}
		\item Assembled and repaired desktops, workstations, and servers.
		\item Tested and debugged hardware peripherals for reuse or recycling.		
		\item Installed Linux OS and other software.
	\end{itemize}
	
\expheader{Microprocessor Design Term Project}{Portland State University}{Portland, Oregon}{3/2013}
	
		\begin{itemize} \setlength{\itemsep}{-4pt}
			\item Implemented simulation of branch predictor, branch target buffer as a term project for Comp. Architecture.
			\item Designed N-way associative cache, return address stack, and fully-associative cache structures in C++	.
			\item Evaluated performance of many design iterations using instruction traces from an unknown processor.
		\end{itemize}	
	
\expheader{Embedded Systems with FPGAs Term Project}{Portland State University}{Portland, Oregon}{12/2012}
	
	\begin{itemize}\setlength{\itemsep}{-4pt}
		\item Designed, assembled, and programmed an autonomous, color-seeking robot controlled by an FPGA-based SoC.  
		\item Implemented computer vision, environmental awareness, and behavioral control using Verilog and ASM.
	\end{itemize}

		
\vspace{-10pt}
% Awards, etc	
\header{Honors}\vspace{3pt}

\expheader{Member and Organizer}{Etta Kappa Nu (HKN)} {IEEE Honors Society}{12/2012 - Current}

	\begin{itemize} \setlength{\itemsep}{-4pt}	
		\item Planed and organized events to promote HKN at Portland State University.
	\end{itemize}	\vspace{-2pt}

\expheader{Scholarship Recipient}{Ford Family Foundation}{Academic Scholarship}{}

	\begin{itemize} \setlength{\itemsep}{-4pt}
		\item Awarded Ford Family Foundation Scholarship for leadership and academic excellence.\hfill\emph{8/2005}
		\item Awarded Graduate Scholarship for academic excellence.\hfill\emph{8/2012}
	\end{itemize}

	
%	
%	\newpage
%	\begin{itemize}
%		\item \NameUrlTextBlock{Intel Professional Recognition}{http://db.tt/0nqPo3Yf} 	%\vspace{-16pt}
%		\begin{itemize}
%		\item Awarded for "excellent creative and technical abilities", after identifying, debugging and repairing a critical functionality in key product in 2013.
%		\end{itemize} \vspace{-6pt}
%		
%		
%		\item \NameUrlTextBlock{Ford Family Foundation Scholarships \emph{(Undergraduate and Graduate)}}{http://www.tfff.org/} \vspace{-16pt}
%		\begin{itemize}
%		\item One of fewer than 100 Scholars inducted into the prestigious Ford Family Foundation Undergraduate Scholarship program in 2005. \vspace{-3pt} 
%		\item Awarded second, additional Graduate Scholarship following academic successes at University of Oregon and Portland State University. \vspace{-3pt}
%		\end{itemize}
%		\item \NameUrlTextBlock{Etta Kappa Nu (HKN)}{http://web.cecs.pdx.edu/~eta/} \vspace{-16pt}
%		\begin{itemize}
%			\item Plan and organize events for IEEE Honors Society. Limited to top 25\% of Department. \vspace{-3pt}
%		\end{itemize}
%
%%		\item \NameUrlTextBlock{Golden Key International Honour Society}{https://www.goldenkey.org/}\vspace{-16pt}
%%		\begin{itemize}
%%			\item Member since 2008\vspace{-3pt}
%%		\end{itemize}
%
%	\end{itemize}
%
%% references
%%\header{References}
%%	\emph{Available upon request.}
%
%% PAGE BREAK
%\newpage
%\fancyhf{}
%









% Projects	
%\header{Academic Projects}
%	\vspace{-15pt}
%	
%	\begin{itemize}
%		\item \NameUrlTextBlock
%		{FPGA-Based Color-Tracking Robot (Verilog, PicoBlaze)}{http://tinyurl.com/color-fpga-bot}\vspace{-16pt}
%		\begin{itemize}
%			\item 1$^{st}$ place in class competition for best term project (SoC design with FPGAs), awarded by course professors. \vspace{-3pt}
%			\item Designed and built autonomous color-seeking robot controlled by SoC in FPGA.  \vspace{-3pt}
%			\item Controlled via custom Verilog code and Picoblaze soft processor in FPGA of Spartan-6E development board
%			\vspace{-3pt} 
%			\item Computer vision via CMUCam4 camera and environmental awareness via proximity/light sensors.\vspace{-3pt}   		
%			\end{itemize}
 
%		\item \NameUrlTextBlock
%		{Branch Predictor/BTB simulation (C)}	{https://github.com/rattboi/flanders_ece486}\vspace{-16pt}
%		\begin{itemize}
%			\item Accuracy of solution ranked within top 3 best designs in Computer Architecture course when tested against traces taken from unknown processor.\vspace{-3pt} 
%			\item Development of branch predictor/branch target buffer simulation in C.\vspace{-3pt} 
%			\item Design of N-way associative cache, Return Address Stack, and Fully-Associative cache simulation blocks in C++			
%		\end{itemize}
%		
%		\item \NameUrlTextBlock
%		{Microprocessor Cache Simulation (Verilog)}{https://github.com/ekrause/0xBEEFA55}\vspace{-16pt}
%		\begin{itemize}
%			\item Source code from this project has been incorporated into Microprocessor Design course materials by professor\vspace{-3pt} 
%			\item Designed and coded split-level L1 data/instruction cache simulation in Verilog\vspace{-3pt}
%			\item Project read in trace data from text file, performed cache simulation, and displayed hit/miss statistics.		\vspace{-3pt}
%		\end{itemize}
%		
%	\end{itemize}
%	\vspace{-4pt}

\end{document}
